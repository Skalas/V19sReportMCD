\lhead{\emph{Introducción}}
\chapter{Introducción}

% repo
Desde su creación en 2003, el Instituto Federal de Acceso a la Información Pública (IFAI ahora INAI) busca garantizar el derecho de acceso de los ciudadanos a la información pública gubernamental (DAI) y a la privacidad de sus datos personales. A la luz de la creación del sistema nacional de transparecncia (SNT) en 2016, se debe conocer de manera integral toda la información que se ha generado en el marco del ejercicio del DAI a nivel nacional para informar el diseño de éste.

Resulta necesario diagnosticar el ejercicio del DAI hasta la fecha para diseñar políticas basadas en evidencia. Para esto, es necesario recopilar la información de los diferentes sistemas internos a través de los cuáles el Instituto facilita al ciudadano el ejercicio del DAI. Igualmente, deben revisarse los indicadores que el INAI genera para evaluar el ejercicio del DAI. Posteriormente, es importante analizar, definir y evaluar indicadores complementarios que permitirán conocer y comparar a nivel nacional la manera en que se ha ejercido el DAI e identificar las variables que al día de hoy es posible extraer con la información recopilada en los registros administrativos del INAI. Este ejercicio diagnóstico requiere de un mapeo de las fuentes de información, de la detección de los vacíos de información \footnote{Por ejemplo, la experiencia del usuario del ejercicio del DAI en cada una de sus etapas.} y de la definición de una metodología que permita evaluar el desarrollo del ejercicio del DAI e identificar áreas de oportunidad de manera oportuna.

A través de este diagnóstico, se emitirán requerimientos para incorporar en las herramientas o plataformas tecnológicas que utilice el INAI, que permitan recopilar información o generar los datos faltantes de manera automatizada y permanente a nivel nacional y, propiciar que este diagnóstico esté disponible, actualizado y en datos abiertos para quienes quisieran consultarlo o explotarlo.

Este ejercicio se realiza en un momento oportuno tras la aprobación de la Ley General de Transparencia \parencite{lgt}, misma que otorga nuevas funciones y atribuciones legales al Instituto. Entre los cambios que incluye la nueva legislación se destacan los siguientes por su efecto en el ejercicio del DAI:

\begin{enumerate}
\item Construcción del \textbf{Sistema Nacional de Transparencia}: permite que el derecho de acceso a la información se ejerza de manera similar en todo el territorio nacional.
\item Aumento de las \textbf{obligaciones de oficio}: aumentan de 17 a 48. Esto implica que, por default, hay mas información disponible de manera pública sin necesidad de que el particular pase por procesos de solicitud de información o por recurso de revisión para ejercer su DAI.
\item Se amplían los \textbf{sujetos obligados}: instituciones, partidos políticos, sindicato, judicial, legislativo, estados y municipios.
\item Toda institución pública y toda persona física o moral que ejerza recursos públicos tiene que reportarlos y transparentarlos.
\item \textbf{Plataforma Nacional de Transparencia}: facilitirá el ejercicio del DAI integrando la funcionalidad del POT, INFOMEX y ZOOM, además de incorporar a los nuevos sujetos obligados y la información requerida de instituciones y personas (físicas o morales) que ejercen recursos públicos.
\end{enumerate}

La nueva ley obliga al ahora INAI a generar nuevos sistemas cuyo objetivo es facilitar el ejercicio del DAI. De manera que se pueda informar la elaboración de estos nuevos sistemas para que cumplan con su objetivo es menester diagnosticar los sistemas que el IFAI proporcionó de 2003 a la fecha para facilitar el ejercicio del derecho. El objetivo de este proyecto es la realización de este diagnóstico. Asimismo, se conceptualizará la manera en la que éste puede ser comunicado a los servidores públicos cuya tarea es la realización de todos los sistemas bajo a nueva LGT. Por último, se identificarán los huecos de información que existen para evaluar el ejercicio del DAI y el efecto que los diferentes sistemas del IFAI tuvieron en su facilitación. Se emitirán recomendaciones para la recopilación de los datos necesarios, a través de los nuevos sistemas del INAI, para que se puedan tener evaluaciones más precisas entorno al ejercicio del derecho y el papel que el órgano garante tiene sobre éste.


\section{Antecedentes}

% tdr final
El Instituto Nacional de Transparencia, Acceso a la Información y Protección de Datos Personales (INAI) es un organismo autónomo encargado de promover y difundir la transparencia, el ejercicio del derecho de acceso a la información y la protección de datos personales.

Adicionalmente, el Reglamento Interior del IFAI establece que la Dirección General de Políticas de Acceso (DGPA) tiene entre sus atribuciones, diseñar y coordinar políticas para impulsar la transparencia y rendición de cuentas. Entre las cuales destaca la de elaborar informes, diagnósticos y proponer recomendaciones para incentivar la mayor difusión de información gubernamental; generar conocimiento sobre los temas relacionados con la transparencia, la rendición de cuentas y otros temas que contribuyan con los objetivos de la Ley y desarrollar estudios y opiniones sobre asuntos de coyuntura y estrategia que impacten en la política de transparencia.

Con la reforma constitucional al artículo sexto en el 2014 y la aprobación de la Ley General de Transparencia y Acceso a la Información Pública, se crea el Sistema Nacional de Transparencia, Acceso a la Información y Protección de Datos Personales (SNT); según el cual se coordinarán, entre otros, los esfuerzos relativos a la  política pública de acceso a la información. La coordinación se realiza entre las distintas instancias gubernamentales e incluye procedimientos, instrumentos y políticas.

Por otra parte, la Ley General establece que la Plataforma Nacional de Transparencia (PNT) es una plataforma electrónica que permite cumplir con los procedimientos, obligaciones y disposiciones señaladas en la Ley General para los sujetos obligados y Organismos garantes, de conformidad con la normatividad que establezca el Sistema, de manera digital e inmediata. La PNT estará conformada por un sistema de solicitudes de acceso a la información, un sistema de gestión de medios de impugnación, un sistema de portales de obligaciones de transparencia, y por último, un sistema de comunicación entre Organismos garantes y sujetos obligados.

Uno de los requisitos de la creación de la PNT es homologar los distintos sistemas informáticos que actualmente utiliza el INAI, al igual que las bases de datos que les corresponden. Dichos sistemas son los siguientes: INFOMEX, ZOOM, Portal de Obligaciones de Transparencia (POT) y Herramienta de Comunicación (HCOM). Actualmente, cada uno de dichos sistemas atiende una función específica y cuenta con bases de datos segmentadas. 

En este sentido, el proyecto estratégico \#MapaDAI, coordinado por la Dirección General de Políticas de Acceso (DGPA) propone la realización de un diagnóstico acerca del estado que guarda el derecho de acceso a la información en México. Dicho diagnóstico podrá conducir a integrar la información de carácter nacional generada por las distintas bases de datos en las entidades federativas y otros sujetos obligados de la LFTAIPG en un sólo repositorio de información nacional.

El propósito del proyecto \#MapaDAI es realizar un análisis estadístico basado en la integración de los datos derivados de los sistemas informáticos del Instituto y de otros organismos garantes y dar a conocer, desde una perspectiva técnica basada en resultados, cómo se ha ejercido el derecho de acceso a la información. 

Este estudio también busca emitir recomendaciones tecnológicas que permitan recopilar información; generar mayores datos de manera automatizada y permanente a nivel nacional; así como establecer recomendaciones para desarrollar una herramienta que permita la visualización de diagnóstico y la descarga de información contenida en el mismo en datos abiertos para quienes quisieran consultarlo o explotarlo.

\section{Situación actual}

% tdr final
Actualmente, el Instituto cuenta con varios sistemas que facilitan a particulares ejercer su DAI: 

\begin{enumerate}
\item INFOMEX es el sistema a través del cuál se gestionan las solicitudes de información.
\item ZOOM permite consultar el histórico de solicitudes realizadas.
\item POT permite la consulta por fracción de los datos que los sujetos obligados deben proporcionar según la \textcite{lft}.
\item H-COM es el sistema interno que utiliza el IFAI para gestionar recursos de revisión.
\end{enumerate}

Estos sistemas atienden a diferentes etapas del ejercicio del DAI y, por ende, no tienen la misma estructura de datos. La información y atributos que se recaban en cada etapa del ejercicio del derecho son distintos. La primera parte de este proyecto se concentra en la recopilación, consolidación y estructuración de la información relevante.

Ahora bien, la mayor complicación para la evaluación estadística del DAI yace en la imposibilidad de mapear de punta a punta el ejercicio del DAI de cada particular. Es decir, es imposible conocer a los usuarios de los sistemas y el camino que deben recorrer para ejercer su derecho. Parte de la restricción a los datos que se pueden recabar está dada por la estructuración de la ley. Por ejemplo, es imposible conocer de manera completa y confiable el perfil demográfico de los usuarios del DAI debido a que la LFT estipula que la información pública de oficio debe estar disponible al público sin necesidad de que el usuario se identifique de manera alguna. Por lo tanto, este tipo de información no se conoce ni se puede conocer. Sin embargo, parte de la información faltante para evaluar el ejercicio del DAI puede ser recabada con la herramienta tecnológica apropiada.

Los datos que serán utilizados en este proyecto se encuentran dispersos en las bases de producción de los diferentes sistemas del INAI (para el ámbito federal) y de los organismos garantes de los otros sujetos obligados en el ámbito estatal. La disparidad en la forma en que están almacenados los datos de los sistemas resulta en una brecha informativa que impide conocer el estado actual en el ejercicio del derecho de acceso a la información y dificulta la formulación de políticas de acceso con base en evidencia empírica. Adicionalmente, las herramientas contienen información para años distintos. 

Como solución a esta problemática, se plantea generar un análisis estadístico de los datos que permita realizar inferencias válidas acerca de los mismos y que contenga elementos de los diversos sistemas informáticos del INAI, permitiendo a su vez recoger la información del ejercicio del derecho de acceso a la información a nivel nacional de forma integral y estructurada. A partir del análisis de inferencia que se realice, se propone elaborar un diagnóstico estadístico sobre el ejercicio y garantía del derecho de acceso a la información desde la creación del INAI a la fecha.   


\subsection{Problemas a solucionar}

% repo
\begin{enumerate}
\item La información de las distintas herramientas que se utilizan actualmente para el ejercicio y la garantía del derecho de acceso a la información está dispersa.
\item Las herramientas tienen distintos años de creación y periodos de implementación y no son interoperables (ejemplo: POT e INFOMEX).
\item Existe información que se estructura manualmente en bases de datos (ejemplo: Los cumplimientos, denuncias y vistas a órganos internos de control de los recursos resueltos por el pleno con alguna instrucción).
\item El INAI, en el marco de la Plataforma Nacional de Transparencia, va a requerir integrar información proporcionada por los sujetos obligados contemplados por la nueva Ley General y órganos garantes estatales.
\end{enumerate}

\section{Objetivos}
\label{sec:objetivos}

% tdr final
El objetivo del proyecto es generar un mapeo de información estadística para hacer un análisis de la información contenida en las diversas herramientas del Instituto y de otros órganos garantes que integran procesos relacionados con el acceso a la información, para que ---a partir de éste--- se conozca cómo se ha ejercido el derecho de acceso a la información en México e identificar áreas de oportunidad. 

Se deberán emitir recomendaciones de mejora a incorporar en las herramientas o plataformas tecnológicas que utilice el INAI, para la recopilación información y generación de mayores datos de manera automatizada. 

De igual manera, se requiere la conceptualización para el desarrollo de una herramienta digital que permita la visualización del diagnóstico y la descarga de información en datos abiertos, para quienes quisieran consultarlo o explotarlo.


\subsection{Objetivo general del proyecto}

% tdr final
Identificar riesgos y áreas de oportunidad en el ejercicio y garantía del derecho de acceso a nivel nacional, para lo cual es necesario mapear la información disponible e identificar la inexistente; derivada de todos los procesos relacionados con el acceso a la información. El proceso de mapeo e identificación de la información permitirá generar un diagnóstico que facilite la toma de decisiones y la elaboración de políticas enfocadas a mejorar el acceso a la información. 


\subsection{Objetivos específicos}

% tdr final
\begin{enumerate}
\item Realizar inferencia estadística sobre el ejercicio y garantía del derecho de acceso a la información para encontrar patrones en el entendimiento del comportamiento del usuario y de los sujetos obligados.
\item Conceptualizar una herramienta de consulta del diagnóstico que recopile y permita el análisis y descarga en datos abiertos de la información sobre el ejercicio y garantía del derecho de acceso a la información a nivel nacional para diseñar políticas basadas en evidencia.
\item Promover, fomentar y difundir la cultura de la transparencia y acceso a la información.
\end{enumerate}

\subsection{Etapas de desarrollo}


\begin{enumerate}
\item Consolidación de información de subsistemas en un Data warehouse (DWH)
\item Construcción de indicadores sobre el ejercicio del DAI.
\item Construcción de una API que permita el acceso a la información recopilada.
\item Visualización de la información recabada
\end{enumerate}
    
\section{Público objetivo}

% tdr final
\begin{enumerate}
\item Público en general.
\item Servidores públicos.
\item Funcionarios de organismos nacionales e internacionales.
\item Medios de comunicación.
\item Organizaciones de la Sociedad Civil.
\item Académicos, investigadores y consultores.
\item Estudiantes.
\end{enumerate}
