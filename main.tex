%% ----------------------------------------------------------------
%% Thesis.tex -- MAIN FILE (the one that you compile with LaTeX)
%% ---------------------------------------------------------------- 

% Set up the document
\documentclass[a4paper, 11pt, oneside]{Thesis}  % Use the "Thesis" style, based on the ECS Thesis style by Steve Gunn
\graphicspath{Figures/}  % Location of the graphics files (set up for graphics to be in PDF format)

% Include any extra LaTeX packages required
% \usepackage[square, numbers, comma, sort&compress]{natbib}  % Use the "Natbib" style for the references in the Bibliography
\usepackage[
backend=biber,
style=alphabetic,
sorting=ynt,
citestyle=authoryear
]{biblatex}
\addbibresource{Bibliography.bib}
\usepackage{csquotes}
\usepackage{enumitem}
\usepackage{verbatim}  % Needed for the "comment" environment to make LaTeX comments
\usepackage{vector}  % Allows "\bvec{}" and "\buvec{}" for "blackboard" style bold vectors in maths
\hypersetup{urlcolor=blue, colorlinks=true}  % Colours hyperlinks in blue, but this can be distracting if there are many links.

\usepackage[utf8]{inputenc}
\usepackage[spanish]{babel}
\usepackage{longtable}

\usepackage{pdfpages}
\usepackage{listings}
\usepackage{float}
%% USA 12.5 CENTIMETROS PARA EL MARGEN!
%% ----------------------------------------------------------------
\begin{document}
\frontmatter      % Begin Roman style (i, ii, iii, iv...) page numbering

% Set up the Title Page
\title  {Thesis Title}
\authors  {\texorpdfstring
            {\href{animalsalvaje@gmail.com}{Andrea Fern\'andez Conde}}
            {\href{escalas5@gmail.com}{Miguel Angel Escalante Serrato}}
            }
\addresses  {\groupname\\\deptname\\\univname}  % Do not change this here, instead these must be set in the "Thesis.cls" file, please look through it instead
\date       {\today}
\subject    {}
\keywords   {}
% % title page commented for now
% \maketitle
%% ----------------------------------------------------------------

\setstretch{1.3}  % It is better to have smaller font and larger line spacing than the other way round

% Define the page headers using the FancyHdr package and set up for one-sided printing
\fancyhead{}  % Clears all page headers and footers
\rhead{\thepage}  % Sets the right side header to show the page number
\lhead{}  % Clears the left side page header

\pagestyle{fancy}  % Finally, use the "fancy" page style to implement the FancyHdr headers

%% ----------------------------------------------------------------
%% % Declaration Page required for the Thesis, your institution may give you a different text to place here
%% \Declaration{

%% \addtocontents{toc}{\vspace{1em}}  % Add a gap in the Contents, for aesthetics

%% I, AUTHOR NAME, declare that this thesis titled, `THESIS TITLE' and the work presented in it are my own. I confirm that:

%% \begin{itemize} 
%% \item[\tiny{$\blacksquare$}] This work was done wholly or mainly while in candidature for a research degree at this University.
 
%% \item[\tiny{$\blacksquare$}] Where any part of this thesis has previously been submitted for a degree or any other qualification at this University or any other institution, this has been clearly stated.
 
%% \item[\tiny{$\blacksquare$}] Where I have consulted the published work of others, this is always clearly attributed.
 
%% \item[\tiny{$\blacksquare$}] Where I have quoted from the work of others, the source is always given. With the exception of such quotations, this thesis is entirely my own work.
 
%% \item[\tiny{$\blacksquare$}] I have acknowledged all main sources of help.
 
%% \item[\tiny{$\blacksquare$}] Where the thesis is based on work done by myself jointly with others, I have made clear exactly what was done by others and what I have contributed myself.
%% \\
%% \end{itemize}
 
 
%% Signed:\\
%% \rule[1em]{25em}{0.5pt}  % This prints a line for the signature
 
%% Date:\\
%% \rule[1em]{25em}{0.5pt}  % This prints a line to write the date
%% }
%% \clearpage  % Declaration ended, now start a new page

%% ----------------------------------------------------------------
%% % The "Funny Quote Page"
%% \pagestyle{empty}  % No headers or footers for the following pages

%% \null\vfill
%% % Now comes the "Funny Quote", written in italics
%% \textit{``Write a funny quote here.''}

%% \begin{flushright}
%% If the quote is taken from someone, their name goes here
%% \end{flushright}

%% \vfill\vfill\vfill\vfill\vfill\vfill\null
%% \clearpage  % Funny Quote page ended, start a new page
%% ----------------------------------------------------------------

%% % The Abstract Page
%% \addtotoc{Resumen}  % Add the "Abstract" page entry to the Contents
%% \abstract{
%% \addtocontents{toc}{\vspace{1em}}  % Add a gap in the Contents, for aesthetics

%% The Thesis Abstract is written here (and usually kept to just this page). The page is kept centered vertically so can expand into the blank space above the title too\ldots

%% }

%% \clearpage  % Abstract ended, start a new page
%% ----------------------------------------------------------------

%% \setstretch{1.3}  % Reset the line-spacing to 1.3 for body text (if it has changed)

%% % The Acknowledgements page, for thanking everyone
%% \acknowledgements{
%% \addtocontents{toc}{\vspace{1em}}  % Add a gap in the Contents, for aesthetics

%% The acknowledgements and the people to thank go here, don't forget to include your project advisor\ldots

%% }
%% \clearpage  % End of the Acknowledgements
%% ----------------------------------------------------------------

\pagestyle{fancy}  %The page style headers have been "empty" all this time, now use the "fancy" headers as defined before to bring them back


%% ----------------------------------------------------------------
\lhead{\emph{\'Indice}}  % Set the left side page header to "Contents"
\tableofcontents  % Write out the Table of Contents

%% ----------------------------------------------------------------
% \lhead{\emph{Figuras}}  % Set the left side page header to "List if Figures"
% \listoffigures  % Write out the List of Figures

%% ----------------------------------------------------------------
% \lhead{\emph{Lista de Tablas}}  % Set the left side page header to "List of Tables"
% \listoftables  % Write out the List of Tables

%% ----------------------------------------------------------------
\setstretch{1.5}  % Set the line spacing to 1.5, this makes the following tables easier to read
\clearpage  % Start a new page
\lhead{\emph{Glosario}}  % Set the left side page header to "Abbreviations"
%\listofsymbols{ll}{}  % Include a list of Abbreviations (a table of two columns)
\chapter{Glosario}
\begin{center}
    \begin{longtable}{p{3cm}p{11.5cm}}
% \textbf{Acronym} & \textbf{W}hat (it) \textbf{S}tands \textbf{F}or \\
\textbf{INAI} & Instituto Nacional de Transparencia, Acceso a la Información y Protección de Datos Personales. \\
\textbf{DAI} & Derecho de Acceso a la Información. \\
\textbf{SNT} & Sistema Nacional de Transparencia, Acceso a la Información y Protección de Datos Personales. \\
\textbf{PNT} & Plataforma Nacional de Transparencia \\
\textbf{DGPA} & Dirección General de Políticas de Acceso. \\
\textbf{Documentación DGPA} & Incluye todos los metadatos, catálogos, documentos metodológicos, comentarios a códigos e instrucciones de reproducción. \\
\textbf{INFOMEX} & Sistema del Gobierno Federal para gestionar solicitudes de información.\\
\textbf{POT} & Portal de Obligaciones de Transparencia.\\
\textbf{H-COM} & Herramienta de comunicación interna del INAI.\\
\textbf{ZOOM} & Buscador de Solicitudes de Información y Recursos de Revisión. Es una herramienta de búsqueda de solicitudes que se han formulado al Gobierno Federal, de las respuestas que se han proporcionado, y de las resoluciones que el INAI emite ante las inconformidades de los ciudadanos respecto a las respuestas que obtienen. \\
\textbf{Diagnóstico \#MapaDAI} & Es un diagnóstico que integra la información de distintos subsistemas informáticos y bases de datos del INAI, así como de todos los demás órganos garantes a nivel nacional. \\
\textbf{Herramienta \#MapaDAI} & Es la herramienta que permite visualizar los resultados arrojados por el Diagnóstico \textbf{ \#MapaDAI}. \\
\textbf{DWH} & Un data warehouse o almacén de datos es una colección de datos orientada a un ámbito específico, en este caso, el ejercicio del DAI, que ayuda a la toma de decisiones en la Organización que la utiliza. Su diseño debe de favorecer el análisis y la divulgación de los datos. \\
\textbf{Data mart} & Es una versión particular de un DWH que tiene como objetivo satisfacer las necesidades de datos de un área específica, está orientado a la consulta de datos.\\
\textbf{ETL} & Extract, transform and load (extraer, transformar y cargar) es el proceso que permite acomodar datos provenientes de múltiples fuentes, hacer todas las transformaciones necesarias para cargarlos en otra base de datos, data mart o data warehouse.\\
\textbf{ORM} & Object-relational mapping (mapeo objeto relacional) es una técnica de programación para pasar de un sistema de tipos en programación orientada a objetos a una base de datos relacional.\\
\textbf{API} & Application Programming Interface (interfaz de programación de aplicaciones) incluye a todas las funciones que ofrece una biblioteca para ser utilizado por otro software. \\
\textbf{ODBC} & Open DataBase Connectivity es un estándar de acceso a una base de datos.\\
\textbf{Server} & Un servidor es una aplicación en ejecución que atiende las peticiones de un cliente y devuelve una respuesta.\\
\textbf{Host} & Un host es una computadora o dispositivo conectado a una red de computadoras. Éste puede ofrecer información, servicios o aplicaciones a otros nodos de la red. \\
\textbf{AWS} & Amazon Web Services es una colección de servicios de web en la nube ofrecidas por Amazon. \\
\textbf{Arquitectura} & A grandes rasgos, se entiende como el plan para integrar, centralizar y mantener los datos de distintas fuentes. \\
\textbf{Tecnologías} & Por tecnologías se entederá únicamente el software a utilizar.\\
\textbf{Flujo de datos} & Se incluye la ingesta de datos, su almacenamiento y los ETL's.\\
\textbf{Script} & Es un archivo de comandos secuenciales, generalmente utilizado en lenguajes interpretados. \\
\textbf{Dueño/administrador} & encagado(s) de los flujos de datos, arquitectura y tecnologías en cada subsistema.\\
\textbf{Protocolos de integración} & se entenderán por éstos los acuerdos con los dueños/administradores de los sistemas para utilizar los ETL's y el DWH construido, conectarlo a su flujo de datos y darle mantenimiento.\\
\textbf{Data wrangling o munging} & Término informal para denotar al proceso manual de convertir o mapear los datos desde un formato bruto (tal cuál se capturan) a uno más conveniente para el consumo de los datos. Se incluye en este término proceso adicionales a los datos como: transformaciones adicionales a estructuras de datos definidas, generación de variables, agregaciones, ordenamiento, entrenamiento de modelos estadísticos, almacenamiento, entre otros. \\
\end{longtable}
\end{center}

%% ----------------------------------------------------------------
% \clearpage  %Start a new page
% \lhead{\emph{Symbols}}  % Set the left side page header to "Symbols"
% \listofnomenclature{lll}  % Include a list of Symbols (a three column table)
% {
% % symbol & name & unit \\
% $a$ & distance & m \\
% $P$ & power & W (Js$^{-1}$) \\
% & & \\ % Gap to separate the Roman symbols from the Greek
% $\omega$ & angular frequency & rads$^{-1}$ \\
% }
%% ----------------------------------------------------------------
% End of the pre-able, contents and lists of things
% Begin the Dedication page

\setstretch{1.3}  % Return the line spacing back to 1.3

\pagestyle{empty}  % Page style needs to be empty for this page
% \dedicatory{For/Dedicated to/To my\ldots}

\addtocontents{toc}{\vspace{2em}}  % Add a gap in the Contents, for aesthetics


%% ----------------------------------------------------------------
\mainmatter	  % Begin normal, numeric (1,2,3...) page numbering
\pagestyle{fancy}  % Return the page headers back to the "fancy" style

% Include the chapters of the thesis, as separate files
% Just uncomment the lines as you write the chapters

\input{Chapters/introduccion} % antescedentes, situacion actual & objetivos

% \item \input{Chapters/marco_conceptual}

% \input{Chapters/pot}

% \input{Chapters/solicitudes}
% \graphicspath{{rgenerated/}}
% \input{rgenerated/infomex}

% \input{Chapters/rr}
% \graphicspath{{rgenerated/}}
% \input{rgenerated/rr}

% \input{Chapters/usuarios}
% \graphicspath{{rgenerated/}}
% \input{rgenerated/perfil_usuario}

% \input{Chapters/ejercicio_del_dai}

% \input{Chapters/recomendaciones}

% \input{Chapters/herramienta}

% \input{Chapters/arquitectura}

%\input{Chapters/Chapter2} % Background Theory 

%\input{Chapters/Chapter3} % Experimental Setup

%\input{Chapters/Chapter4} % Experiment 1

%\input{Chapters/Chapter5} % Experiment 2

%\input{Chapters/Chapter6} % Results and Discussion

%\input{Chapters/Chapter7} % Conclusion

%% ----------------------------------------------------------------
% Now begin the Appendices, including them as separate files

\addtocontents{toc}{\vspace{2em}} % Add a gap in the Contents, for aesthetics

\appendix % Cue to tell LaTeX that the following 'chapters' are Appendices

% \input{Appendices/AppendixA}	% Appendix Title

%\input{Appendices/AppendixB} % Appendix Title

%\input{Appendices/AppendixC} % Appendix Title

\addtocontents{toc}{\vspace{2em}}  % Add a gap in the Contents, for aesthetics
\backmatter

%% ----------------------------------------------------------------
\label{Bibliography}
\lhead{\emph{Bibliography}}  % Change the left side page header to "Bibliography"
%\bibliographystyle{unsrtnat}  % Use the "unsrtnat" BibTeX style for formatting the Bibliography
%\bibliography{Bibliography}  % The references (bibliography) information are stored in the file named "Bibliography.bib"
\printbibliography

\end{document}  % The End
%% ----------------------------------------------------------------