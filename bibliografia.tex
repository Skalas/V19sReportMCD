\begin{thebibliography}{1}



\bibitem{cnn} CNN Español {\em 19 de septiembre, la fecha fatídica que dejó huella entre los mexicanos.} 19 septiembre 2017: \url{https://cnnespanol.cnn.com/2019/09/19/cientos-de-muertos-miles-de-damnificados-y-millones-de-dolares-en-perdidas-asi-fue-el-terremoto-del-19s-en-mexico/}. Fecha de consulta: 20 de abril de 2021.

\bibitem{codd} Codd, E. F. {\em A Relational Model of Data for Large Shared Data Banks.} 1970:  Communications of the ACM. 13 (6): 377–387. \url{doi:10.1145/362384.362685}.

\bibitem{coord} Ann Majchrzak, Sirkka L. Jarvenpaa, and Andrea B. Hollingshead. {\em Coordinating Expertise Among Emergent Groups
Responding to Disasters.} 2007: Organization Science 18 (1) 147-161 \url{https://doi.org/10.1287/orsc.1060.0228}.

\bibitem{mymap} Ikeda, Yoshiyasu, Yosuke Yoshioka, and Yasuhiko Kitamura. {\em Intercultural collaboration support system using disaster safety map and machine translation.} 2010: Culture and Computing 100-112. Springer, Berlin, Heidelberg, 2010.

\bibitem{ap19s} Animal Político {\em Lo que el \#19S nos dejó: las víctimas, daños y damnificados en México.} 19 de octubre, 2017:
\url{https://www.animalpolitico.com/2017/10/cifras-oficiales-sismo-19s/}. Fecha de consulta: 20 de abril de 2021.

\bibitem{telcom19s} El Economista {\em ¿Por qué fallaron las líneas telefónicas tras el sismo del 19 de septiembre de 2017?} 20 de septiembre de 2017: \url{https://www.eleconomista.com.mx/empresas/Por-que-fallaron-las-lineas-telefonicas-tras-el-sismo-del-19-de-septiembre-de-2017-20170920-0091.html}. Fecha de consulta: 20 de abril de 2021.

\bibitem{ift} Instituto Federal de Telecomunicaciones. {\em A 72 horas del sismo, 98\% de las redes públicas de telecomunicaciones se encuentran en funcionamiento.} 22 de septiembre de 2017:
\url{http://www.ift.org.mx/comunicacion-y-medios/comunicados-ift/es/72-horas-del-sismo-98-de-las-redes-publicas-de-telecomunicaciones-se-encuentran-en-funcionamiento}. Fecha de consulta: 20 de abril de 2021.

\bibitem{flood} Department for Environment, Food and Rural Affairs, Flood Risk Management Division, {\em Spontaneous volunteers: Involving citizens in the response and recovery to flood emergencies.} London. \url{http://randd.defra.gov.uk/Document.aspx?Document=13013_FD2666_FinalReport_SpontaneousVolunteers.pdf} Consultado el 23 de abril del 2021.

\bibitem{emergentgroups} Twigg J, Mosel I. {\em Emergent groups and spontaneous volunteers in urban disaster response. Environment and Urbanization.}  2017;29(2):443-458. \url{doi:10.1177/0956247817721413}.

\bibitem{harvardhuman} Harvard Humanitarian Initiative (2011) {\em Disaster relief 2.0: the future of information sharing in humanitarian emergencies.} UN Foundation and Vodafone Foundation Technology Partnership, Washington, D.C. and Berkshire.

\bibitem{crowdsourced} Hunt, Amelia, and Doug Specht. {\em Crowdsourced mapping in crisis zones: collaboration, organisation and impact.} Journal of International Humanitarian Action 4.1 (2019): 1-11.

\bibitem{networkshum} Chernobrov, Dmitry. {\em Digital volunteer networks and humanitarian crisis reporting.} Digital Journalism 6.7 (2018): 928-944.

\bibitem{bigdatahum} Meier, Patrick. 2015a. {\em Digital Humanitarians: How Big Data is Changing the Face of Humanitarian Response.} Boca Raton, FL: CRC Press.

\bibitem{ushahidi} Ushahidi Team {\em About Ushahidi} 8 Septiembre 2019:\url{https://docs.ushahidi.com/ushahidi-platform-user-manual/about-ushahidi}. Fecha de consulta: 24 de mayo de 2021.

\bibitem{guardianusha} Simon Jeffery {\em Ushahidi: crowdmapping collective that exposed Kenyan election killings.} 7 de abril de 2011. \url{https://www.theguardian.com/news/blog/2011/apr/07/ushahidi-crowdmap-kenya-violence-hague} Fecha de consulta: 24 de mayo de 2021.

\bibitem{costushahidi} Ushahidi Team {\em Pricing} \url{https://www.ushahidi.com/pricing} Fecha de Consulta 24 de mayo 2021.

\bibitem{previouscost} Ushahidi Team {\em Setting up a deployment} \url{https://docs.ushahidi.com/ushahidi-platform-user-manual/2.-setting-up-a-deployment} Fecha de Consulta 24 de mayo 2021.

\bibitem{osmwiki} OpenStreetMap Wiki contributors, {\em Main Page,} OpenStreetMap Wiki, \url{https://wiki.openstreetmap.org/w/index.php?title=Main_Page&oldid=2013332} , Fecha de Consulta 25 de mayo 2021.

\bibitem{chilton} Chilton S, {\em Crowdsourcing is radically changing the geodata landscape: case study of openstreetmap.}, 2009, Proceedings of the 24th international cartographic conference.

  \bibitem{meier2012} Patrick Meier (2012): {\em Crisis Mapping in Action: How Open Source Software and Global Volunteer Networks Are Changing the World, One Map at a Time}, Journal of Map \& Geography Libraries: Advances in Geospatial Information, Collections \& Archives, 8:2, 89-100

  \bibitem{hotosmorg} Humanitarian OpenstreetMap Team, {\em What we do} Humanitarian OpenStreetMap website, \url{https://www.hotosm.org/what-we-do} Fecha de consulta: 25 de abril 2021.

  \bibitem{sorden2014} Soden, Robert, and Leysia Palen. {\em From crowdsourced mapping to community mapping: The post-earthquake work of OpenStreetMap Haiti.} COOP 2014-Proceedings of the 11th International Conference on the Design of Cooperative Systems, 27-30 May 2014, Nice (France). Springer, Cham, 2014.
  \bibitem{batty2010} Batty Peter  {\em OpenStreetMap in Haiti Part 1.} \url{http://www.youtube.com/watch?v=PyMTKABxaw4}. Fecha de Consulta 30 de abril 2021.

  \bibitem{mora2011} Mora, Fernando. {\em Innovating in the midst of crisis: A case study of Ushahidi.} Submitted for publication to SAGE Convergence Journal 3.5 (2011): 231-245.

  \bibitem{meier2010} Norheim-Hagtun, Ida, and Patrick Meier. {\em Crowdsourcing for crisis mapping in Haiti.} Innovations: Technology, Governance, Globalization 5.4 (2010): 81-89.

  \bibitem{Aiko2014} Takazawa, Aiko. {\em Action at a Distance: How do Ordinary People Self-organize Humanitarian Efforts Remotely and Collaboratively?.} iConference 2014 Proceedings (2014).
  \bibitem{Alberto2018}   Alberto, Yolanda, et al. {\em Reconnaissance of the 2017 Puebla, Mexico earthquake.} Soils and foundations 58.5 (2018): 1073-1092.
  \bibitem{Verde1991} Verde, Roberto Villa. {\em Explanation for the numerous upper floor collapses during the 1985 Mexico City earthquake.} Earthquake engineering \& structural dynamics 20.3 (1991): 223-241.
\end{thebibliography}
